% !TeX root = these.tex
\chapter{La base de données}
la base de données se veut, délibérément, extrêmement simple/minimal.  Le but n'est pas de gérer l'information mais plutôt de la stocker.

Par exemple, on ne se soucie pas de donnée propre à l'année, la section, l'implantation, etc.,  Nous nous focalisons uniquement sur les Attributions. Les informations qui les composent se doivent d'être séparé en différente partie (par exemple, les professeurs, les différents groupes, etc.

L'objectif n'étant pas de faire une base de données relationnelle dans les règles de l'art (genre éviter les répétitions dans les tables, y a un mot pour ça mais je parviens pas à le retrouver, il est 5h du matin ^^), mais d'avoir les données rassemblées en un point fixe (comme les classes (Année, Section, groupe) (même si il y a des répétitions des données (genre section T)) et de pouvoir accéder aux données de la manière la plus rapide qu'il soit. Nous avons donc opté pour ce système de stockage, évitant les cout imputer au passage entre les différentes tables pour récupérer l'ensemble des informations dont nous avions besoins, et permettant de garder une logique sur ce que doit représenter une attribution.

%Xavier, peut tu retravailler cette partie ;-)
une autre table très particulière est les activityState, qui correspond à l'états d'un carton (idéalement à un instant donné), cad, si ce carton est placé, où et qd.  Le but est de pouvoir offrir des retour en arrière ou une annalyse des mouvements effectué, comme par exemple en selectionnant un carton et voir les états précédent de se dit cartons (mais cela n'est pas fait dans la version actuel du programme) ainsi que de priviligier l'ajout à lecture (car il faudrait faire une recherche et un update à chaque pose de cartons), et également permetre de régler des litiges pouvant survenir (utilisation synchrone du programme par exemple).  Il rend également la gestion des différentes "Instances" bcp plus simples.  L'inconvénient, c'est qu'il faut faire une recherche un peu plus longue lors du chargement du projet car il faut selectionner le nernier état du carton. Egalement, il est possible que cette table pourrait atteindre de grosse tailles, et il est necessaire, une fois le projet cloturé, de supprimer les données inutiles. 

Les "instances" sont un set d'états, xxx

pour la gestions des contraintes, il est également necessaire de les sauvegarder, et seulement un set non exostif de contraintes est possibles d'etre enregistré (et traité dans notre programme). Néanmoins, ce set est relativement large et permet de "nommer" la plus part des contraintes.  Il s'agit principalement de "préférences".  Préférences qui pourront être "hard" ou "soft" (un indiquateur est possible pour y mettre plus de grénulosité, mais il n'est actuelement pas pris en compte par notre solveur).  préférence applicable sur un Moment (jour/period) ou sur un(des) local(aux).  préférences dont la sources peuvent être prof, cours, groupe, local.

Avec ca, il est possible de 


