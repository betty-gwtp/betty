\chapter*{Conclusion}

Dans le présent rapport, nous avons vu comment répondre au problème réel et récurrent de la création d'horaires dans le contexte académique. Pour répondre à cette problématique, nous avons dû effectuer un large travail sur trois points en particuliers ; la programmation par contraintes, une solution client/serveur et, enfin, un travail collaboratif en équipe.
\newline
\indent
Pour la programmation par contraintes, nous 

-- on conclut dans le foin :) --\\

->  point cool perso: utilisre des nouvelles technologies \\

->  ce projet nous tien à coeur \\
non, il n'est pas encore completement utilisable dans le sans que nous voudrions.
Il est possible de créer horraire, mais les avantage qu'offrent l'outils informatique ne sont eux pas fonctionnel. Notre solution, actuelement offre donc moins d'avantage que le faire en version papier.  Nous croyons cependant bcp dans notre programme et pensons sincèrement, que mme si c'était pas les les plus simple, les décisions que nous avons prise peuvent permettre énormément de choses.  \\\\

-> nous avons été supris par la dificulté (dans le sans qu'un petit scout est
surpris par la nuit :p ), et il à été très frustrant de devoir faire face à
d'autre problèmes qu'on avait pas imaginé et qui nous ont dévié de nos objectif
premier.  Heureusement ce fut des problème interessant qui nous ont bcp appris,
que ca soit sur la gestion du code, gestion de notre ``équipe'' (fair ce tfe à
deux était une exelente expérience. ca nous à permit de fournir du code plus
propre, car relu, des idées plus aboutie, \ldots Notre gestion, et la grosseur
de la tâche ainsi que sa ``segmentation'' nous aurais permi d'éfectuer ce tfe
avec 2fois plus de participant, sans qu'ont ne se ``marche sur les pieds'',..)
