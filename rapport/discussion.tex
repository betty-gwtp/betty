% !TeX root = these.tex
\chapter{Discussions}
\section{Choix concernant la performance}
\section{Choix concernant la sécurité}

Du fait de nos choix de conception, tel que gwt pour générer le java-script, ou
hibernate pour abstraire la base de donnée, ou le choix des rpc pour la
commnuniquation client-serveur,..  Nous pensons avoir créé une application ``de
base'', très sécure, et qu'il serait tout à fait envisageable de faire tourner notre
application sur un serveur externe.
(xxx) xss, csrf, sha256, check (basique) de la complexité du mot de passe, ainsi
qu'un catcha très basique aussi pour eviter des bots.

La secu du coté serveur à également été pris en compte, mais, logiquement, en
moins poussé (quoi que).  A savoir: l'application tourne sur une debian *stable*
(à savoir Squeeze), n'ayant que peu de services installé.
Tomcat, notre conteneur d'applet, est lui aussi à jours, et en version stable,
il ne bénificie d'aucun droit root, ainsi donc, (contrairement à une autre
application tournant sur Windows par exemple), il n'a pas le droit d'emmetre sur
le port 80. Pour garder l'application safe et peformante, une redirection de
son port d'origine est effectué par netfiller sur le port 80.

Nous avons pas poussé plus loins la sécu du coté serveur, étant donnée qu'elle
est destiné à fonctionner sur d'autre machine, mais en situation réel, il serait
jusdicieu de mettre en place un liens https.  En situation réel, nous
préconisons xxx (parfeu, strapping, vm,..)

N'étant pas infaillble, la mise à jour (très aisée pour notre type
d'application), est nous semple -t-il, une part importante dans la sécurité du
système. (cela probablement accentué par la mise de notre code source sous une
license ouverte (gpl3) )
