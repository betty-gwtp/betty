\section{Choix de conception}
Nos objectifs sont d'avoir un programme simple d'utilisation et interactif utilisant les données de l'EPHEC.
Ses fonctions seront :\\
\begin{itemize}
      \item de permettre la création d'horaire par l'enregistrement de cours dans des plages horaires choisies manuellement,
      \item d'aider aux choix de ces plages horaires en fonction des données déjà enregistrées,
      \item de permettre le calcul d'un horaire en fonction de contraintes données par l'utilisateur.\\
\end{itemize}

Les données contenues dans les serveurs concernent: les cours, les classes, le nombre d'élèves, les professeurs,
 le nombre de sièges par classe, ...\\
Les contraintes données par l'utilisateur sont: le nombre limité de sièges par classe, les absences des professeurs, ... \\

Pour la simplicité d'utilisation, notre réflexion nous a poussé vers une application internet.
Les avantages sont multiples: aucune installation donc aucun problème de compatibilité avec les différents OS,
 accessible depuis n'importe quel ordinateur, ... 
Une application internet est composée de plusieurs parties: un côté serveur et un côté client.
Le côté client nous permet d'avoir une interface interface, le côté serveur permet l'accès,
 le traitement (càd le calcul des horaires) et l'enregistrement des données.\\

%*ce serait sympa d'avoir un schéma super basique*\\

Cette section du rapport s'attelle à décrire et justifier plus en détail les technologies et choix de conception utilisés. 


  \subsection{Application internet}
  Comme nous l'avons dit précedemment, le fait de travailler sous la forme d'application web donne de nombreux avantages 
  mais pose certains problèmes, notamment concernant la stabilité de la connexion internet.
  Pour éviter ces écueils et améliorer le côté réactif de notre application, nous avons cherché à travailler principalement en local storage.
  La connexion n'est donc nécessaire que lors de la première connexion au programme afin de télécharger les données du serveur et lors de l'enregistrement.\\

  Parler du truc asynchrone que j'ai rien compris :D\\

  accessibilité aux serveurs tout le temps??? \\

  \subsection{Le côté client}

  La partie client est celle utilisée par la personne qui va réaliser l'horaire. 
  Nos objectifs sont d'avoir une interface graphique userfriend et interactive qui permette de créer l'horaire.
  C'est également dans cette partie que va se trouver l'aide à la création des horaires.
  Cette aide est simplement constituée d'un code couleur permettant de savoir si une plage horaire est déjà occupée ou non. \\

    \subsubsection{GWT}
    GWT est l'outil qui nous permet de résoudre plusieurs problèmes :
    réaliser une interface web interactive asynchrone, permettre l'enregistrement local des données, ....

  \subsection{La base de donnée}
  La base de donnée représente la base de donnée existante actuellement à l'EPHEC. Pour l'instant, nous avons utilisé un autre serveur.
  Mais l'acquisition des données ...

    \subsubsection{Hibernate}

  \subsection{Le côté serveur}
  La partie serveur est composée de la base de donnée et d'une partie solveur.
  Le solveur est simplement une partie calculatoire qui permettra d'optimiser la création de l'horaire.

    \subsubsection{Java}
    \subsubsection{TomCat}


  \subsection{Discussion entre client et serveur}
    \subsubsection{RPC}
