\section{Analyse de la problématique et outils existant}
\subsection{La programmation par contraintes}
Avant tout, il est nécessaire de bien saisir la problématique et en quoi celle-ci peut être utile à notre travail. Nous avons du faire un analyse approfondie sur ce qu'était une contraintes ainsi que la programmation de celles-ci. Il a fallu ensuite analyser les différents outils nous permettant de "programmer par contraintes" et ensuite essayer de les implémenter dans un logiciel d'aide à la création d'horaire que nous avons du créer au préalable.\\
\\
%D'abord, pourquoi on a fait ça et pourquoi ce TFE. En quoi c'est utilite pr notre travail
Madame Gillet, Directrice de l'établissement Ephec à Louvain-la-Neuve, établit l'horaire a chaque nouveau quadrimestre de l'année scolaire. l'établissement d'un horaire doit se faire sous certaines contraintes et en fonction de désidératas remis par les professeurs. Le nombre de locaux informatique est limité, certains professeurs sont dit "externe" à l'Ephec, ceux-ci doivent se voir attribué un horaire particulier, certain cours se donne dans des locaux externes à l'Ephec et ne sont donc pas disponible à chaque période de cours. Toute ces contraintes, si elle ne sont pas informatisée doivent prise en compte par la personne établissant l'horaire qui doit réaliser un "vrai casse tête chinois" afin d'avoir un horaire le plus adéquat possible.\\
\\
La programmation par contrainte permet de résoudre de manière automatique cette problématique, facilitant ainsi la tâche qui incombe à la personne en charge de l'élaboration d'un horaire. Nous allons d'abord définir ce qu'est une contraintes de manière plus théorique pour bien saisir la problématique et la provenance de celle-ci. Nous discuterons ensuite autour des différentes solutions existantes nous permettant de réaliser cela.

%Parler de notre rencontre avec Pierre bidule je sais plus quoi
%je sais plus ce qu'on avais dit d'autre