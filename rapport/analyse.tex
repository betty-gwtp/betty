% !TeX root = these.tex

\chapter{Analyse de la problématique et outils existants}
%Avant tout, il est nécessaire de bien saisir la problématique et en quoi celle-ci peut être utile à notre travail. Nous avons du faire un analyse approfondie sur ce qu'était une contraintes ainsi que la programmation de celles-ci. Il a fallu ensuite analyser les différents outils nous permettant de "programmer par contraintes" et ensuite essayer de les implémenter dans un logiciel d'aide à la création d'horaire que nous avons du créer au préalable.\\
%\newline
%\indent
%D'abord, pourquoi on a fait ça et pourquoi ce TFE. En quoi c'est utilite pr notre travail
%Madame Gillet, Directrice de l'établissement EPHEC à Louvain-la-Neuve, établit l'horaire a chaque nouveau quadrimestre de l'année scolaire. l'établissement d'un horaire doit se faire sous certaines contraintes et en fonction de désidératas remis par les professeurs. Le nombre de locaux informatique est limité, certains professeurs sont dit "externe" à l'EPHEC, ceux-ci doivent se voir attribué un horaire particulier, certain cours se donne dans des locaux externes à l'EPHEC et ne sont donc pas disponible à chaque période de cours. Toute ces contraintes, si elle ne sont pas informatisée doivent prise en compte par la personne établissant l'horaire qui doit réaliser un "vrai casse tête chinois" afin d'avoir un horaire le plus adéquat possible.\\
%\newline
%\indent
%La programmation par contrainte permet de résoudre de manière automatique cette problématique, facilitant ainsi la tâche qui incombe à la personne en charge de l'élaboration d'un horaire. Nous allons d'abord définir ce qu'est une contraintes de manière plus théorique pour bien saisir la problématique et la provenance de celle-ci. Nous discuterons ensuite autour des différentes solutions existantes nous permettant de réaliser cela.

%Parler de notre rencontre avec Pierre bidule je sais plus quoi
%je sais plus ce qu'on avais dit d'autre

Afin de mieux comprendre la problématique liée à l'élaboration d'un horaire, nous avons tout d'abord pris rendez-vous avec Madame \textsc{Gillet}, directrice de l'EPHEC. Madame \textsc{Gillet} est la personne en charge de l'élaboration des horaires pour la partie Louvain-la-Neuve de l'EPHEC. Nous avons parcouru ensemble les différents problèmes et les différents types de contraintes auxquelles nous devrons faire face. Cette première analyse du problème avait pour but de nous orienter sur le type de contraintes intervenant dans notre travail.
\newline
\indent
Dans un premier temps, cette présente section, théorisera la notion de contrainte. Dans un deuxième temps, seront présentées les différentes contraintes rencontrées dans l'analyse du problème. 

\section{Une contrainte en théorie}
Dans le cadre d'un horaire, nous devons faire face à des contraintes sur des domaines finis. Nous allons illustrer ce concept comme suite :
\begin{center}
$P=(X,D,C)$
\end{center}

$X$ étant nos variables. Celles-ci seront représentées par les attributions\footnote{Les attributions sont les différents cours avec leurs propriétés intrinsèques. Dans le cadre de ce rapport, nous désignerons ces cours par l'appellation de cartons. Cette notion sera explicitée plus loin dans le présent rapport.} dans lesquelles nous retrouverons les éléments suivants :
\begin{itemize}
\item nom du professeur
\item le cours
\item la classe
\end{itemize}
Nous pouvons l'écrire sous la forme $Vn$ ou $V$ correspond à la valeur du carton, et n sont numéro.\\
\newline
\indent
	$D$ correspond au domaine. Ici, il s'agit d'effectuer un horaire. Notre domaine sera le nombre de locaux disponibles multiplié par le nombre de périodes. Dans le cadre du cours du jour à l'EPHEC de Louvain-la-Neuve, nous avons 6 périodes par jour, durant 5 jours. Un local est donc disponible à 6 moments de la journée durant 5 jours/semaine. Il faut donc multiplier le nombre de local par 30. Ce domaine s'écrit donc sous la forme $Nl x Np$.
\newline
\indent
	$C$ étant nos contraintes. Il existe différents types de contraintes. Dans la section suivante, nous en énumérerons certaines pour bien se rendre compte du nombre important de celles-ci et dans quel moment elles interviennent.\\
\newline
\indent
Pour régler ce problème de contraintes, il est avantageux de prendre en compte certaines heuristiques ou métaheuristiques. Nous avons défini le problème rencontré à l'EPHEC comme  étant de type NP-Complet.
\newline
\indent
La problématique de la programmation par contrainte est que les algorithmes de résolution tentent à trouver une solution \texttt{optimale} au problème. Si une contrainte ne peut être prise en compte lors de cette tentative, nous n'obtiendrons pas de résultat. Pour ce faire, il est nécessaire d'appliquer un filtrage de ce qui ne peut être pris en compte. Ce \enquote{mécanisme} s'appelle la \texttt{propagation de contraintes}.
\newline
\indent
La propagation de contraintes a pour objectif d'appliquer un filtrage sur les attributions posant problèmes et de les retirer du pool de cartons\footnote{Le pool de cartons présente l'ensemble des cours qu'il est possible de positionner dans le semainier. Cette notion de pool de cartons sera vue dans le prochain chapitre.} pris en charge lors de la résolution de l'horaire. Il est nécessaire de prendre en compte le temps que le filtrage va prendre afin de garantir une meilleure rapidité de la résolution.

\section{Les types de contraintes}

Pour se rendre compte de l'ensemble des contraintes à prendre en compte lors de l'établissement de l'horaire, nous en avons énuméré un maximum dont voici la liste :\\

\begin{table}[h!]
\begin{minipage}[t]{.3\linewidth}
    \begin{tabular}{|l|}
    \hline
\textbf{Locaux}\\
\hline
\hline
Matériels: (transparent, audio, Projecteur)\\
\hline
Type de local (sale info, auditoire, TP)\\
\hline
Nombres de places disponibles\\
\hline
Situation géographique \\
\hline
Attribution aux classes d'élèves\\
\hline
Marie Curie Ou Place des Sciences\\
\hline   
    \end{tabular}
\end{minipage}
\hfill
\begin{minipage}[t]{.4\linewidth}
    \begin{tabular}{|l|}
    \hline
\textbf{Professeurs}\\
\hline
\hline
Desiderata\\
\hline
Statut (externe, plein temps, ...)\\
\hline
Local où se donne le cours\\
\hline
Exigences matérielles\\
\hline
Pas d'enfants dans la classe\\
\hline
Temps Louvain-la-Neuve ou Bruxelles\\
\hline
    \end{tabular}
\end{minipage}
%\caption{\texttt{datatypes properties} appelées pour la description des types \textit{MusicGenre}, \textit{RecordLabel} et \textit{Settlement}}
\caption{Contraintes sur les locaux et les professeurs}
\label{locaux_prof}
\end{table}

\begin{table}[h!]
\begin{minipage}[t]{.3\linewidth}
\begin{tabular}{|l|}
\hline 
\textbf{Cours}\\
\hline
\hline
Étalement dans le temps\\
\hline
Cours qui se suivent obligatoirement (ex : TP)\\
\hline
Cours qui ne doivent pas se suivre (ex: AN+NDLS)\\
\hline
Groupe ou demi-groupe\\
\hline
Nombre d'heures total\\
\hline
Hebdomadaire ou bimensuel\\
\hline
\end{tabular}
\end{minipage}
\hfill
\begin{minipage}[t]{.4\linewidth}
\begin{tabular}{|l|}
\hline 
\textbf{Attribution (ou carton)}\\
\hline
\hline
Doit pouvoir ignorer les contraintes\\
\hline
Doit pouvoir ajouter des contraintes\\
\hline
\end{tabular}
\end{minipage}
%\caption{\texttt{datatypes properties} appelées pour la description des types \textit{MusicGenre}, \textit{RecordLabel} et \textit{Settlement}}
\caption{Contraintes sur les cours et les attributions}
\label{cours_attribution}
\end{table}

%\begin{center}
%\begin{tabular}{|l|}
%\hline 
%\textbf{Cours}\\
%\hline
%\hline
%Étalement dans le temps\\
%\hline
%Cours qui se suivent obligatoirement (ex : TP)\\
%\hline
%Cours qui ne doivent pas se suivre (ex: AN+NDLS)\\
%\hline
%Groupe ou demi-groupe\\
%\hline
%Nombre d'heures total\\
%\hline
%Hebdomadaire ou bimensuel\\
%\hline
%\end{tabular}
%\end{center}
%\bigskip
%
%
%\begin{center}
%\begin{tabular}{|l|}
%\hline 
%\textbf{Attribution (ou carton)}\\
%\hline
%\hline
%Doit pouvoir ignorer les contraintes\\
%\hline
%Doit pouvoir ajouter des contraintes\\
%\hline
%\end{tabular}
%\end{center}

\newpage

\begin{table}[!h]
\begin{center}
\begin{tabular}{|p{\linewidth}|}
\hline 
\textbf{Contraintes générales}\\
\hline
\hline
Éviter les heures de fourches\\
\hline
Si possible, garder les élèves dans un même local et faire déplacer le professeur\\
\hline
Ordre sur les contraintes (poids sur celle-ci) afin de favoriser un externe plutôt qu'un interne\\
\hline
Attribuer les locaux informatiques à un jour donné pour une option\\
\hline
Séparer eu deux semestres\\
\hline
Prendre en compte les jours de congé\\
\hline
Directrice doit définir demi-classe, groupe, etc. pour un cours\\
\hline
Il doit être possible de reporter un cours\\
\hline
Il doit être possible de définir une pause\\
\hline
\end{tabular}
\end{center}
\caption{Contraintes générales}
\end{table}
\bigskip


Pour des améliorations futures, d'autres contraintes pourront éventuellement être prises en charge comme :\\

\begin{table}[h!]
\begin{minipage}[t]{.3\linewidth}
\begin{tabular}{|l|}
\hline 
\textbf{Étudiants}\\
\hline
\hline
Répartition en fonction du sexe (sauf TI)\\
\hline
Répartition en fonction de la provenance\\
\hline
Possibilité de changer de classe\\
\hline
\end{tabular}
\end{minipage}
\hfill
\begin{minipage}[t]{.4\linewidth}
\begin{tabular}{|l|}
\hline 
\textbf{Classes}\\
\hline
\hline
Nombre d'étudiants maximum\\
\hline
Locaux adaptés\\
\hline
\end{tabular}
\end{minipage}

\caption{Contraintes sur les étudiants et les classes}
\label{etudiant_classes}
\end{table}
\bigskip



En conclusion, nous pouvons voir que le nombre de contraintes à prendre en compte pose un réel problème quand elles doivent être traitées par un humain. Les possibilités, les préférences sur certains types de contraintes devant être absolument prises en compte ou pas, relève d'une réflexion de haut niveau. Les ordinateurs "réfléchissant" de manière plus "mathématique" son en théorie plus apte à répondre de ce genre de problème.

\section{Analyse des applications existantes}
Nous avons fait également une petite étude sur les applications déjà existantes permettant d'élaborer des horaires avec contraintes. Nous pouvons citer \texttt{EDT}, programme très repandu dans les écoles, \texttt{aSc} Horaire possédant une partie automatisée et manuelle pour la création d'horaire ainsi que \texttt{Université time tabling} (UniTime). Cette dernière est une application très prometteuse, mais ne propose que l'établissement d'horaire automatique. Aucune façon manuelle de faire n'est proposée. Il utilise sa propre librairie pour résoudre la problématique des contraintes.

\section{La programmation par contraintes}
Nous avons abordé les contraintes de manière théorique, les avons énumérées et cité les applications déjà existantes, qu'en est-il de l'application de cette théorie? Pour pouvoir programmer par contraintes il existe plusieurs libraire. Certaines propriétaires et d'autres open source. Notre travail ayant pour objectif d'être libre et open source, nous nous sommes intéressés à ces dernières et en avons testé certaines :\\
\begin{itemize}
\item GeCode
\item Google OR tools
\item Python Constraint
\end{itemize}
\bigskip

Python constraint, constitue une bonne approche de la problématique, le code étant clair et concis. Celui-ci sera le plus parlant pour illustrer la programmation par contraintes de manière plus concrète.

\begin{figure}[!h]
\begin{lstlisting}[frame=single]
 from constraint import *
 problem = Problem()
 problem.addVariable("a", [1,2,3])
 problem.addVariable("b", [4,5,6])
 problem.getSolutions()
[{'a': 3, 'b': 6}, {'a': 3, 'b': 5}, {'a': 3, 'b': 4},
 {'a': 2, 'b': 6}, {'a': 2, 'b': 5}, {'a': 2, 'b': 4},
 {'a': 1, 'b': 6}, {'a': 1, 'b': 5}, {'a': 1, 'b': 4}]

 problem.addConstraint(lambda a, b: a*2 == b,
                          ("a", "b"))
 problem.getSolutions()
[{'a': 3, 'b': 6}, {'a': 2, 'b': 4}]

 problem = Problem()
 problem.addVariables(["a", "b"], [1, 2, 3])
 problem.addConstraint(AllDifferentConstraint())
 problem.getSolutions()
[{'a': 3, 'b': 2}, {'a': 3, 'b': 1}, {'a': 2, 'b': 3},
 {'a': 2, 'b': 1}, {'a': 1, 'b': 2}, {'a': 1, 'b': 3}]
\end{lstlisting}
\caption{Code python}
\end{figure}

\textbf{Debug : Mettre une explication du code ici}\\

Dans le cadre de notre solution, nous avons utilisé \texttt{GeCode} et \texttt{Google OR tools}. Ces deux outils sont très puissants et nous ont permis de répondre dans une certaine mesure aux contraintes explicitées.

\section{Rencontre}
Dans le cadre de ce travail, nous avons eu l'occasion de rencontrer monsieur Pierre \textsc{Schaus} qui a effectué une thèse de doctorat à l'UCL sur la problématique des contraintes. Notre rencontre nous a permis de mieux cerner la problématique.
\newline
\indent
Suite à cet entretien, nous avons pu mieux nous rendre compte de la problématique et des vigilances à prendre lors de la programmation par contraintes. Le solveur ne se contente pas d'être une librairie dans laquelle il suffirait d'appeler des fonctions pour résoudre un problème et avoir un résultat. Les choses doivent se faire petit à petit, car si le solveur ne trouve pas de résultat, il n'est pas possible de savoir pourquoi celui-ci n'a pu en trouver un. 
\newline
\indent
Il faut donc y rentrer nos contraintes pas à pas. Certaines contraintes étant plus importantes que d'autre, il serait également utile de pouvoir leur mettre un poids. Ceci n'étant pas possible\footnote{L'implémentation actuelle proposée ne permet pas de mettre un poids d'importance sur les contraintes.}, il a fallu trouver un autre moyen. Monsieur Pierre \textsc{Schaus} nous a alors conseillé d'utiliser une approche par hiérarchisation de nos contraintes. Cette hiérarchie a été exécutée en utilisant un arbre de contraintes. Celles-ci seront ensuite rentrées dans le solveur suivant leurs importances. Monsieur \textsc{Schaus} étant entrain d'élaborer ses propres librairies, nous à montrer quelques exemple concret de l'utilisation de celle-ci et de la marche à suivre.\\
\newline
\indent
Il nous a ensuite fourni cette librairie, cette dernière n'étant pas encore distribuée actuellement\footnote{Mais elle sera disponible gratuitement et distribuée en open source sous peu.}. Cependant, cette dernière n'était pas adaptée à notre problème précis\footnote{Cette librairie n'est pas adaptée à un problème NP-Complet car il est impossible d'y ajouter des heuristiques.}. En conséquence de quoi, nous nous sommes tournés, dans un deuxième temps, vers  \texttt{GeCode} et \texttt{Google OR tools} pour faire quelques tests et élaborer un premier prototype.

\section{Décision}

Cependant, à la fin de l'année, pour le prototype final, nous nous sommes tournés vers la librairie \texttt{uniTime} et avons exploré de manière plus approfondie ce que l'application propose. Nous nous sommes basés sur cette dernière pour l'élaboration de notre solveur. En effet, les possibilités que cette librairie offre correspondent parfaitement aux contraintes de type horaire. Nous parlerons plus en détail de cette librairie dans la suite de ce travail.



