
% !TeX root = these.tex

\chapter{Présentation de l'application}
Notre application, surnommée Betty\footnote{Brillant Ephec Time Tabling for You}, a pour but d'être un outils permettant d'élaborer un horaire de façon plus conviviale en comparaison avec la façon de procédé utilisée à l'Ephec. Voici comment l'application ce présente de manière générale, et les fonctionnalités qu'elle propose dans son état actuel.

\section{Connexion et Inscription}

La première page de l'application propose à l'utilisateur d'entrer son nom d'utilisateur et son mot de passe. Le mot de passe étant crypté en SHA-256\footnote{Secure Hash Algorithm}, dans la base de données. La page propose également de vous inscrire via la flèche en haut à droite de la fenêtre.\\
\\
Lors de l'inscription, nous invitons l'utilisateur à entrer sont nom d'utilisateur, mot de passe et email. Les informations entrées sont soumises à des vérifications d'usage, le maximum de ces vérifications étant effectuées du coté client afin de réduire l'échange avec le serveur.\\
\\
Pour cette partie, quelques améliorations peuvent être apportées tel que le changement de mot de passe ou la récupération de celui-ci par envoi de mail.

\section{Page des projets}
La page est ordonnée de façon à avoir toujours le dernier projet créé en haut de la liste. Un projet est présenté de la manière la plus simple possible afin de ne pas perdre l'utilisateur. Ce dernier à la possibilité de créer un nouveau projet (option également disponible via le menu), Choisir le quadrimestre à élaborer et de supprimer un projet. L'application devrait dans une prochaine mise à jour proposer des options sur celui-ci tel que le partage de projets entre plusieurs utilisateurs.\\
\\
Lors de la création d'un nouveau projet, celui-ci doit être nommé et contenir les fichiers nécessaires à l'élaboration de l'horaire. Ces fichiers se présentent sous la forme d'un .xls contenant pour le premier, la liste des attributions de chaque cours, ce fichier est le résultat d'une requête SQL et nous a été fournis par l'Ephec. Le deuxième représente la liste des locaux disponibles, ce fichier n'existant pas en tant que tel, à initialement été créé par Madame Vroman, Professeur à l'Ephec, dans le cadre du "projet horaire" du cours de langage avancé de programmation de deuxième année. Nous y avons ajouté des informations pouvant être prisent en compte par le solveur, et permettant de facilité l'établissement manuel d'un horaire.\\
\\
Une fois le projet créer et le quadrimestre sélectionné, l'utilisateur est redirigé vers la page principale de l'application dans laquelle l'horaire pourra être créé.

\section{Page principale}
La page principale se présente comme suit:\\

\begin{itemize}	
	
	\item[-] nord de la page, nous trouvons les différents filtres et options disponibles tel que:\\
	\begin{itemize}
		\item[•] Card filter
		\item[•] Sélection de la grille à afficher
		\item[•] Un panneau regroupant les différentes instances du projet (forme de sous projet)\\
	\end{itemize}
	\item[-] À l'ouest les cartons créés sur base du fichier des attributions\\
	\item[-] Au centre, le semainier où les cartons pourrons venir se glisser\\
	\item[-] À l'est, un panneau de notifications permettant d'avoir un suivit des différentes actions faite par l'utilisateur.
\end{itemize}
\subsection{Les attributions}
La mise en forme des cartons se base sur le design de ceux actuellement employé à l'Ephec lors de l'établissement manuelle de l'horaire. Ceux-ci comporte le/les classe(s) assignée(s) à un cours donné par un professeur.\\
\\
A chaque carton est assigné un ensemble de locaux où pourront se donner le cours. Par exemple, un carton de type informatique serra assigné uniquement aux locaux de type informatique. Lorsque le carton est déposé, le solveur client lui assigne un local disponible parmi cette liste, et nous pouvons voir apparaitre le nom du local en bas à droite du carton.
\subsection{Semainier}
Nous affichons dans le semainier, les informations relatives à la personne, classe ou au local choisi via le filtre prévu à cet effet. La grille se colorie en fonction du carton qui est sélectionné. Un code de couleurs a été mis en place permettant de distinguer si un carton peut être placé ou pas. Par exemple, si l'on se trouve dans la vue d'une classe et que l'on prend un carton d'une autre classe, toute la grille se coloriera en rouge, montrant à l'utilisateur que celui-ci ne peut pas être placé.\\
\\
Nous distinguons trois groupes de couleurs, vert, orange et rouge. Ces groupes sont eux même subdivisés en trois catégories: couleur claire, normale et foncée. Lorsqu'on colorie la grille horaire, il arrive parfois qu'un carton puisse être placé à plusieurs endroits, les uns plus avantageux que d'autres pour la suite de l'établissement de l'horaire, il est donc nécessaire de pouvoir dire à l'utilisateur que le carton peut être placé, mais l'orienter sur un choix plus adéquat. Ce code de couleurs est utilisé dans une version limité pour le moment.
\subsection{Notifications}
Les notifications sont un support pour l'utilisateur. Lorsque celui-ci effectue une action comme supprimer/ajouter un carton il est nécessaire de savoir si l'action c'est effectuée correctement. A la place d'un popup intrusif, nous avons opté pour ce système, signalant à l'utilisateur de manière plus douce l'état d'actions qui ne peuvent être vue via une interface graphique.\\
\\
Nous distinguons ici deux couleurs différentes, une couleur se fondant au thème général de l'application, lorsque tout c'est effectué correctement, et une couleur rouge pour signaler un problème. Ce systèmes est limité et pourra être amélioré dans le futur.
\subsection{Les filtres}
Les différents filtres permettent de faciliter la création de l'horaire de façon manuelle. Si nous voulons créer l'horaire d'un professeur en particulier, avoir les cartons de tous les professeurs rendraient la tâche plus lourde à l'utilisateur. L'utilisation de ce filtre permet d'avoir une meilleure vue sur ce qui doit être placé.\\
De même, il est possible d'afficher/masquer les cartons déjà placé. Lorsqu'on filtre les cartons d'une classe, et que tous ces cartons sont placés, ils ne font théoriquement plus parti de la liste des cartons et donc l'utilisateur n'a pas la possibilité de savoir si la dite classe possède des cartons. Ce système favorise aussi la vue d'ensemble sur ce qui a déjà ou non été placé.\\
\\
Une dernière option est de pouvoir automatiquement switcher sur la grille horaire correspondant au filtre mis sur les cartons. Si cette option est sélectionné, et que nous sélectionnons la classe 3TL2 dans le card filter, nous supposons ici que l'objectif est de placé les attributions de cette classe, la grille horaire des 3TL2 est alors affichée.\\
\\
L'application à donc été pensé afin de minimalité au maximum les tâches devant être effectuées par l'utilisateur et de lui offrir des outils de filtrage avancé.
\subsection{Les instances}
Le panneau d'instance permet, au sein d'un même projet, de créer un ensemble de sous projets. L'objectif principal de ce panneau réside dans l'utilisation du solveur. Celui-ci sera lancé dans une nouvelle instance, permettant à l'utilisateur de continuer sont horaire manuelle dans une autre sans être bloqué. Une fois la résolution finie, l'utilisateur peut naviguer entre les différentes instances afin de voir les différents résultats obtenu. \\
\\
C'est ici que l'implémentation des instances y trouve sa principale utilité, des améliorations futures pourrons être appliquées, par exemple pouvoir comparer deux instances entre elle.
%Cette partie est à améliorer et à modifier!
\subsection{Le solveur}
Pour lancer le solveur, nous devons aller dans le menu project, solveur, solve. Une fenêtre s'ouvre permettant à l'utilisateur de sélectionner l'instance dans laquelle celui-ci doit effectuer la résolution. Une fois le solveur lancé, une notification apparait à l'utilisateur lui disant que celui-ci est exécuté. à la fin de tentative de résolution, l'utilisateur reçoit une notification lui spécifiant que le solveur à fini son travail et si tout c'est déroulé correctement. Le solveur, dans sa version actuel, fonctionne parfaitement mais ne propose aucune options de configuration et fonctionne sur des petits projets. 
\subsection{Mémoire cache}
L'application utilise la mémoire cache du navigateur pour stocker les données, ainsi nous minimalisons les requêtes vers le serveur. Ceci pourrait être également exploité pour pouvoir travailler sur l'application sans connexion internet. Les données nécessaire à l'établissement de l'horaire étant stocker dans la partie cliente.
