% !TeX root = these.tex

\chapter{Justification des choix technologiques}

%************************ CLIENT SERVEUR *********************************
\section{Client/Serveur}
Notre application se base sur une architecture client serveur pour les multiples
avantages que celle-ci apporte.\\
Premièrement, l'EPHEC étant un établissement possédant plusieurs sites, une application client serveur peut permettre d'avoir un centre de données
 commun.
 De cette manière, un horaire établi à Louvain-la-Neuve pourra être pris en
 compte lors de l'établissement d'un horaire à Bruxelles.\\
L'avantage de partager la charge de travail, les grosses parties (solveur, actions lourdes, etc.) étant
effectuées sur le serveur, permet que l'élaboration de l'horaire puisse se
faire sur des machines possédants peu de ressources. Par exemple, il sera
possible d'établir l'horaire sur une tablette ou encore sur un Smartphone.\\
\newline
\indent
Grâce à ce type d'architecture, l'application n'est pas dépendante du système
d'exploitation mis en place ni de l'ordinateur. L'horaire peut être débuté sur
une machine Windows, pour ensuite être continué sur une tablette.
Les mises à jour de l'application sont aussi complètement transparentes pour
l'utilisateur final. Pas besoin de télécharger et d'installer les mises à jour
comme sur un logiciel orienté desktop.\\
De même, en comparaison toujours avec une application desktop, s'il survient un
problème avec la station de travail, il est garanti de pouvoir retrouver ses
données dans leurs entières intégralités, et l'horaire peut continuer à être établi
sur une autre station. Les serveurs offrent de nombreux avantages (duplication des
données, séparation sur plusieurs sites,…) qu'un ordinateur en panne ne peut offrir.\\
\newline
\indent
Outre cet aspect de facilité pour la partie cliente, il est aussi très facile
d'administrer l'application. Celle-ci étant portable et facile d'installation.
Pas besoin de connaissances approfondies ou de configurations spéciales du
serveur. Il suffit d'installer un serveur\footnote{un serveur permettant
de faire tourner une application web java comme: Tomcat, Jetty, etc.} et d'y
mettre l'application\footnote{sous forme de '.WAR'} dans le bon dossier.
De même, la base de données peu être complètement indépendante de l'application
et peu se trouver sur un serveur externe. Ainsi, elle peut par exemple être interne à l'EPHEC et l'application peut se trouver sur un serveur public.
Le type de SGBD\footnote{Système de Gestion de Base de Données} utilisé a peu d'importance, et il n'est pas nécessaire de créer
la base de données au préalable. L'application se charge de la créer d'elle-même
grâce à l'utilisation de l'outil Hibernate. Celui-ci utilise JDBC\footnote{Java Data Base Connectivity}, Hibernate étant une couche supérieure. Il faut cependant noter certains points pour cette partie. Télécharger le driver\footnote{disponible sur http://developers.sun.com/product/jdbc/drivers/} JDBC par celui correspondant à la base de données (Oracle en recense actuellement 221), modifier le fichier de configuration d'Hibernate et créer une base de données nommée betty avec un utilisateur ayant les droits d'administration sur cette base de données. En dehors de ces 3 points, l'application ne nécessite aucune configuration spécifique. Nous avons fait différents tests, avec du mySQL ainsi que du PostgreSQL sur des bases de données vierge, et nous n'avons pas rencontré de problèmes de création.\\
\newline
\indent
% peut être pas mettre ce paragraphe dans le rapport
%Nous avons choisi le langage java d'une part car nous sommes familiariser avec la programmation orienté objet.

%************************ JAVA *********************************
\section{Java}
%Cette partie est plutôt pertinente pour justifier le choix du langage Java au profit d'un autre
Le choix du langage Java s'est fait instinctivement. Celui-ci est très présent
dans le domaine du développement en entreprise. Le langage java permet de bien
structurer son programme,il fait preuve d'une certaine rigueur, de robustesse
et offre la possibilité d'utiliser des variables typées statiques. Nous avons pu
tirer avantage de ces derniers points pour élaborer notre application.\\
\newline
\indent
Nous avions, dans une première approche, l'intention de faire cette
application en Python. Ce Langage offre beaucoup de possibilités et est aussi adapté
au type d'application que nous voulions élaborer. Il possède une structure
favorisant les bonnes pratiques, ce point étant particulièrement important pour
nous. Le langage Java s'est fait plus instinctivement. Tout d'abord, le solveur est codé en java. Nous étions partis dans l'idée de faire du
binding grâce à l'utilisation de SOAP\footnote{Simple Object Access Protocol} entre le Python et le solveur, mais
dans un souci de clarté du code, nous avons préféré rester dans le même type de
langage. L'arrivée de GWT (que nous verrons dans le point suivant) nous à aussi
conforté dans ce choix.
\newline
\indent
Il existe plusieurs types de serveur d'application java. Nous utilisons un
serveur Tomcat, écrit lui-même en java et étant multiplateformes. Ainsi
l'application peut tourner sur n'importe quel type d'architecture.


%************************ JAVA EE *********************************
% Cette partie ne me semble pas utile ici. On parle déjà du java, on dira après
% que c'est du java EE quand on parlera du serveur

%\section{Java EE}
%Pour établir notre logiciel, nous avons donc utiliser du java, sous ça forme
%entreprise édition. Ceci étant nécessaire pour l'élaboration du code partie
%serveur. l'utilisation de Google Web Toolkit nous à imposer cette utilisation,
%et nous avons du 

%************************ Google web Toolkit *********************************
\section{Google Web Toolkit (GWT)}

Nous avons choisi de travailler avec GWT pour les nombreux avantages\footnote{cfr. Chapitre 3.1.1 - GWT} que celui-ci apporte. GWT nous permet de coder la partie cliente de l'application en Java, et celui-ci génère le code JS\footnote{JavaScript} correspondant. Le code généré par GWT peut être adapté aux différents navigateurs les plus répandus à ce jour tels que Chrome, Firefox, Internet Explorer, etc.
\newline
\indent
La partie serveur étant réalisée en java, il est plus facile pour le développeur de créer son application dans un langage unique. GWT utilise du java EE du coté serveur, langage très puissant et ayant déjà fait ces preuves de robustesse puisque celui-ci est très répandu dans le monde professionnel. Google web toolkit propose également l'utilisation d'outils comme Jin and Guice qui feront l'objet d'un autre point. Comme sont nom l'indique, GWT est une vrai boite à outils.


%****************************** Local Storage *********************************
\section{HTML5, CSS3}
Nous avons utilisé les dernières normes de ces langages web. GWT permettant d'utiliser ceux-ci, nous avons décidé d'utiliser certaines fonctionnalités intéressantes qu'elle propose, comme l'utilisation d'un local storage, venu avec le HTML5, permettant de stocker les données coté client. Le CSS3, quand est à lui, est utilisé pour le rendu graphique de l'application.

%******************************** Hibernate ***********************************
\section{Hibernate}
Pour la communication avec la base de données, nous utilisons l'outil Hibernate. Nous ne rentrerons pas dans les détails de cet outil, celui-ci fera l'objet d'un autre point. Nous nous limiterons à souligner que Hibernate est un outil performant permettant de représenter les tables de base de données en objets et facilitant ainsi l'utilisation des données. Il est une couche supérieure à JDBC permettant de communiquer avec une base de données. Il peut être utilisé avec n'importe quel type de SGBD et permet la création automatique des tables. Il possède donc de nombreux avantages et est aussi présent en entreprise.

%************************ Solveur *********************************
\section{Solveur}

Après analyse des différentes libraires, nous nous sommes orientés sur les
librairies \textit{Unitime}. Celles-ci étant beaucoup plus adaptées à nos besoins, qui sont de
pouvoir gérer des contraintes sous la forme de désidératas. Elle sont écrites en java et sont orientées pour la problématique de la création d'horaires d'un établissement scolaire. A défaut des autres librairies étant plus orientées contraintes et non désidératas, celle-ci propose des options de configurations correspondant à nos besoins et aux besoins d'un établissement tel que l'EPHEC. Nous l'avons pris uniquement pour des raisons de performance et de correspondance à ce qui doit être fait, et non pas par facilité d'utilisation ou d'apprentissage.

%************************** GitHub *********************************
\section{GitHub}
Puisque nous faisons ce travail en équipe, il est nécessaire de pouvoir avoir un suivi de ce que chacun de nous fait, de pouvoir fusionner nos travaux et de garder une trace des différentes version en cas de bugs éventuels. Nous avons choisi GitHub pour gérer notre projet, celui-ci étant simple d'utilisation, très performant, gratuit et permettant surtout de fusionner les différents codes écrits, au sein d'une même page, de manière indépendante et intelligente.

