\section{Nos Choix}

%************************ CLIENT SERVEUR *********************************
\subsection{Client/Serveur}
Nos choix ce sont portés sur une architecture client serveur pour les multiples
avantages que celle-ci apporte.\\
Premièrement l'Ephec étant un établissement possédant plusieurs sites,
 une application client serveur peut permettre d'avoir un centre de données
 commun.
 De cette manière, un horaire établi à Louvain-la-Neuve pourra être pris en
 compte lors de l'établissement d'un horaire à Bruxelles.\\
L'avantage de partager la charge de travail, la grosse partie (solveur,…) étant
effectuée sur le serveur, permet que l'élaboraiton de l'horaire puissent se
faire sur des machines possédant peu de ressource. Par exemple, il sera
possible d'établir l'horaire sur une tablette ou encore sur un Smartphone.\\
\\
Grâce à ce type d'architecture, l'application n'est pas dépendante du système
d'exploitation mis en place ni de l'ordinateur. L'horaire peut être débuté sur
une machine Windows, pour ensuite être continuer sur une tablette.
Les mises à jour de l'application sont aussi complètement transparentes pour
l'utilisateur final. Pas besoin de télécharger et d'installer les mises à jour
comme sur un logiciel orienté desktop.\\
De même, en comparaison toujours avec une application desktop, si il survient un
problème avec la station de travail, il est garantis de pouvoir retrouver ces
données dans sont entière intégralité, et l'horaire peut continuer à être établi
sur une autre station. Les serveurs offre bon nombre d'avantage (duplication des
données,…) qu'un ordinateur en pane ne peut offrir (même dans le cas d'une
écrite sur deux disques différents).\\
\\
Outre cet aspect de facilité pour la partie cliente, il est aussi très facile
d'administrer l'application. Celle-ci étant portable et facile d'installation.
Pas besoins de connaissance approfondie, ou de configurations spéciales du
serveur. Il suffit d'installer un serveur\footnote{un serveur permettant
de faire tourner une application web java comme: tomcat, jetty, ect…} et d'y
mettre l'application\footnote{sous forme de '.war'} dans le bon dossier.
De même, la base de données peu être complètement indépendante de l'application
et peu se trouver sur un serveur externe.
Le type de SGBD utilisé à peut d'importance, et il n'est pas nécessaire de créer
la base de données au préalable. L'application se charge de la créer d'elle-même
grâce à l'utilisation de l'outil Hibernate.

% peut être pas mettre ce paragraphe dans le rapport
%Nous avons choisi le langage java d'une part car nous sommes familiariser avec la programmation orienté objet.

%************************ JAVA *********************************
\subsection{Java}
%Cette partie est plutôt pertinente pour justifier le choix du langage Java au profit d'un autre
Le choix du langage Java c'est fait instinctivement. Celui-ci est très présent
dans le domaine du développement en entreprise. Le langage java permet de bien
structurer sont programme,il fait preuve d'une certaine rigueur, de robustesse
et offre la possibilité d'utiliser des variables typé statique. Nous avons pu
tirer avantages de ces derniers points pour établir notre application.\\
Nous avions, dans une première approche, l'intention de faire cette
application en Python. Ce Langage offre beaucoup de possibilité est aussi adapté
au type d'application que nous voulions élaborer. Il possède une structure
favorisant les bonnes pratiques, ce point étant particulièrement important pour
nous. Le langage java c'est fait plus instinctivement. Tout d'abord, les
librairies du solveur sont en java. Nous étions partis dans l'idée de faire du
binding grace à l'utiliation de SOAP entre le python et le solveur, mais
dans un souci de clarté du code, nous avons préféré rester dans le même type de
langage. L'arrivée de GWT (que nous verrons dans le point suivant) nous à aussi
conforté dans ce choix.
Il existe plusieurs types de serveur d'application java. Nous utilisons un
serveur Tomcat, écrit lui-même en java et étant multiplateforme. Ainsi
l'application peut tourner sur n'importe quel type d'architecture.


%************************ JAVA EE *********************************
% Cette partie ne me semble pas utile ici. On parle déjà du java, on dira après
% que c'est du java EE quand on parlera du serveur

%\subsection{Java EE}
%Pour établir notre logiciel, nous avons donc utiliser du java, sous ça forme
%entreprise édition. Ceci étant nécessaire pour l'élaboration du code partie
%serveur. l'utilisation de Google Web Toolkit nous à imposer cette utilisation,
%et nous avons du 

%************************ Google web Toolkit *********************************
\subsection{Google Web Toolkit (GWT)}

Nous avons ensuite choisi de travailler avec GWT pour les nombreux avantages
(chapitre sur GWT) que celui-ci apporte. Dans un souci de portabilité de
l'application sur les différents navigateurs existant le code javascript étant
généré automatiquement par GWT, mais aussi l'accès à celle-ci en utilisant les
dernières technologies offertes par les Smartphones et tablettes, cet outils est
plus qu'efficace et est d'une aide très précieuse. La partie cliente étant
écrite en java comme la partie serveur.

%************************ Solveur *********************************
\subsection{Solveur}

Après analyse des différentes libraires, nous nous sommes orientez sur la
librairie xxx. Celle-ci étant beaucoup plus adapté à nos besoins, qui sont de
pouvoir gérer des contraintes sous la forme de désidérata. 
