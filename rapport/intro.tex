\section{Introduction}
%étapes traditionnellement dans l’introduction :
%• l’accroche, 1) capter l’attention du lecteur
%• la problématique 2) délimiter le sujet
%• l’annonce du plan (très claire) 3) annoncer le plan

Vous en avez marre de voir vos secrétaires souffler et déprimer en septembre en pensant déjà aux cloches
 qu'elles vont avoir en manipulant leurs cartons d'horaire? 
Ce temps-là est révolu. Grâce à notre aide informatique à la création des horaires, elles vont enfin être libérées de ce joug!\\

L'objectif de notre mémoire est d'aider à la création des horaires de l'EPHEC.
Dans cet objectif, nous avons décidé de créer une application internet permettant d'accéder
aux données présentes sur les serveurs de l'école.\\

Cette application est créée pour être utilisée au sein de l'école, c'est pourquoi il faut un 
outil adapté à l'EPHEC simple d'utilisation et interactif. \\

Ce rapport décrira la méthodologie utilisée et les choix de conception faits.
Par la suite, nous décrirons plus profondément le programme et sa structure avant de clore sur une discussion sur les possibilités du programme. 



