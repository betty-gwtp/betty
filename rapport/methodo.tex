% !TeX root = these.tex
\chapter{Méthodologie de conception}
Un programme informatique, comme n'importe quel projet, nécessite une bonne gestion pour pouvoir assurer son bon développement. 
D'une part, il faut faire face aux difficultés liées au travail d'équipe, d'autre part, il faut gérer le projet en tant que tel.

  \section{Git pour le développement collaboratif}
  La création d'un logiciel en équipe implique la gestion de cette équipe ainsi que le partage et l'échange des
  informations entre chaque membre de celle-ci.
  
  L'équipe doit travailler de façon coordonnée. En effet, elle doit s'échanger l'état
  d'avancement du travail ainsi que les résultats obtenus. Dans cette visée, nous nous sommes tournés vers le programme de gestion de versions Git développé par Torvald et, notamment, utilisé dans le cadre du développement du noyau Linux.\\
\newline
\indent
  Git est un logiciel de gestion de version décentralisé. Il permet de
  travailler à plusieurs sur un même projet. Il gère lui-même l'évolution du
  contenu en fusionnant les changements sans perte d'information. En outre, il garde
en mémoire toutes les versions du code. De plus, il est libre et
  adapté aux grands projets.\\
  \newline
  \indent
  Afin de permettre un stockage de notre code, nous avons de nous tourner vers un dépôt GitHub\footnote{GitHub est présenté précédemment.}.
  
  

  \section{Logiciel de suivi de problème}
  
  En plus de reposer sur Git, GitHub offre un logiciel de suivi de problème, aussi appelé  bugtracker. Un logiciel de suivi de problème est en quelque sorte un journal des problèmes classés par type.  C'est un outil particulièrement utile pour le développement en équipe. Il donne un aperçu clair des bugs et de leurs états de résolution.

  \section{Méthode de travail}
Pour ce travail, il a été important d'avoir une méthode rigoureuse. Nous avons donc scindé le travail en plusieurs parties, sous forme de blocs à développer.\\
\newline
\indent
Pour chacun de ces blocs, nous avons noté les problèmes de conceptions et, ensuite, nous avons analysé comment les résoudre. Par exemple,  la difficulté d'implémentation du système d'échanges asynchrones ou, encore, l'absence de support, par GWT, du filtre de cartons ont dû être analysées et ont été l'objet de solution respective. \\
\newline
\indent
Nous avons suivi la méthode RUP\footnote{Rational Unified Process} basée sur UP, définissant un moyen de travailler sur des applications de type orienté objet. Cette dernière fournit de nombreux avantages comme le rôle de chaque acteur du projet, le cycle de vie de l'application, etc. Nous nous sommes donc basés sur cette méthode pour élaborer notre travail dans de bonnes conditions.
\newline
\indent
Nos besoins étant de trouver une méthode de travail en équipe et d'avoir une bonne gestion de projet, nous ne rentrerons pas dans les détails de cette méthode. Toutefois, nous pouvons dire qu'elle nous à été bénéfique pour l'élaboration de notre application. Certains aspects de cette méthode comme la gestion de projets ont pu être utilisés via GitHub et son bugtracker. \\
\newline
\indent
Nous avons conceptualisé notre application dans une optique d'extension. Dans cet objectif, rien n'a été fait statiquement afin de garantir l'évolution future de ce projet. C'est pour ces raisons que certains choix on été effectués, nous rendant la charge de travail plus lourde mais possédant des avantages non négligeables pour la bonne évolution de l'application. L'étendue et le potentiel d'un tel outil-logiciel sont grands, surtout pour un établissement scolaire. Par conséquent, nous nous sommes donc confortés dans cette optique.

\section{Communication}

Une bonne communication interne a dû être assurée afin de pouvoir prendre des décisions sur certaines parties du projet. Bien qu'ayant défini la part de travail de chacun, il est important de bien comprendre ce qui a été fait par l'autre. Il faut aussi pouvoir tirer avantage d'un travail à deux, surtout dans le cadre de réflexion concernant la méthode à utiliser. Par exemple, les choix devant être établis en priorité pour élaborer certaines parties de l'application ou pour régler les bugs, etc. \\
\newline
\indent
Nous avons utilisé des logiciels de VOIP et nous  nous sommes réunis régulièrement afin de discuter des différents points. Ainsi, il était plus aisé de parler des problèmes rencontrés et de trouver, ensemble, une solution à ces problèmes. Certaines tâches ont donc été effectuées séparément et d'autres conjointement.

\section{Apports périphériques}
Nous avons, dans le cadre de nos cours à l'EPHEC, eu l'occasion d'utiliser certaines méthodes permettant de bien clarifier un programme telle que la méthode QQOQCP\footnote{Quoi? Qui? Où? Quand? Comment? Pourquoi?}. Nous avons donc tiré parti de cet apprentissage pour introduire la problématique de notre travail. 
\newline
\indent
Ainsi, en reposant sur les différentes méthodologies existantes, nous avons beaucoup appris sur les problèmes pouvant intervenir dans l'élaboration d'une application (que ce soit web ou desktop). Il est bien entendu impératif de se rendre compte de ces différents points pour pouvoir avoir une cohérence au sein d'une équipe. Il est fréquent pour un développeur de devoir travailler en équipe dans une entreprise. Ainsi, cette approche nous donne l'opportunité d'être mieux préparés au monde professionnel.



    
%        * mettre ici la façon dont vous avez \enquote{fait} le pgm* \\
%    
%    cad comment vous avez fait en pratique : on a ciblé nos besoins (citer les besoins) et nous nous sommes
%    dirigé vers le type de conception/méthodologie MACHIN(*mettre une référence*) qui consiste à faire de l'essai %erreur et être à la bourre 
%sans savoir où on va ni n'ou on vient. MAIS, comme on est pas des poulets, on a fait un planning. Le bugtracker nous a %également été très utile.
%
%Ce paragraphe n'est pas nécessaire, mais dans l'optique où votre promoteur aimera savoir ce que vous avez glandé %pendant l'année, c'est ici qu'il 
%faut lui prouver que 1> vous aviez une bonne méthodologie de travail (important) 2> vous avez pas glandé = pensé à tel %ou tel problème futur, telle ou telle option future, tel truc plus compliqué à faire mais qui permet ceci ou cela.
%Attention, QUE du blabla, rien de technique (vu qu'on l'explique que après)