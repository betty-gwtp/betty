% !TeX root = these.tex
\chapter{Méthodologie de conception}
Un programme informatique, comme n'importe quel projet, nécessite une bonne gestion pour assurer un bon déroulement de ce projet.
D'une part, il faut faire face aux difficultés liées au travail d'équipe, d'autre part, il faut gérer le projet en tant que tel.

  \section{GIT et partage des données}
  La création d'un logiciel en équipe implique la gestion de cette équipe ainsi que le partage et l'échange des
  informations entre chaque membre de celle-ci.
  L'équipe doit travailler de façon coordonnée, elle doit s'échanger l'état
  d'avancement du travail ainsi que les résultats obtenus. Pour ce faire, il faut
  de bons moyens de communication, ou dans notre cas, UN bon moyen de communication: GIT\cite{git}. \\

  GIT est un logiciel de gestion de version décentralisé. Il permet de
  travailler à plusieurs sur un même projet. Il gère lui-même l'évolution du
  contenu en fusionnant les changements sans perte d'information. Il garde
  également en mémoire toutes les versions du code. Il est libre, gratuit et
  particulièrement facile à utiliser. Le code se trouve sur un dépôt internet, nous avons choisi GitHub\cite{github}.\\

  \section{Logiciel de suivi de problème}
  En plus de cette gestion de versions, GitHub offre un logiciel de suivi de problème, c'est-à-dire, un bugtracker.
  Un logiciel de suivi de problème est en quelque sorte un journal des problèmes classés par type.
  C'est un outil particulièrement utile pour le développement en équipe. Il donne un aperçu clair des bugs et de leurs états.

  \section{Méthodologie de conception}
    * mettre ici la façon dont vous avez \enquote{fait} le pgm* \\
    
    cad comment vous avez fait en pratique : on a ciblé nos besoins (citer les besoins) et nous nous sommes
    dirigé vers le type de conception/méthodologie MACHIN(*mettre une référence*) qui consiste à faire de l'essai erreur et être à la bourre 
sans savoir où on va ni n'ou on vient. MAIS, comme on est pas des poulets, on a fait un planning. Le bugtracker nous a également été très utile.

Ce paragraphe n'est pas nécessaire, mais dans l'optique où votre promoteur aimera savoir ce que vous avez glandé pendant l'année, c'est ici qu'il 
faut lui prouver que 1> vous aviez une bonne méthodologie de travail (important) 2> vous avez pas glandé = pensé à tel ou tel problème futur, telle ou telle option future, tel truc plus compliqué à faire mais qui permet ceci ou cela.
Attention, QUE du blabla, rien de technique (vu qu'on l'explique que après)