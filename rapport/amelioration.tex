% !TeX root = these.tex
\chapter{Perspectives}

\section{Possibilités d'amélioration du programme}

\subsection{Mode comparaison}
Cette partie étant un de nos objectifs principaux. Il est question de pouvoir créer à la
\enquote{volée} les colonnes représentant plusieurs professeurs, locaux ou classes, tout
 en gardant, pour les lignes, les périodes. Malheureusement nous n'avons pas eu le temps de l'implémenter. Un autre problème fondamental ayant retardé son élaboration est qu'il n'est  possible de rendre une région ne faisant pas partie du DOM (typiquement rajouté 
en JS, donc d'une manière non-statique) comme étant une région \enquote{dropable}.
Cela n'étant pas possible, la solution idéale nous est encore inconnue. Il faudrait probablement rendre l'entièreté du tableau comme \enquote{dropable} car actuellement uniquement les cases du tableau sont dropables.

\subsection{Solveur}
Ce fut également un point assez frustrant, étant donné nos recheches 
sur le sujet, et surtout la trouvaille de cette librairie ``miraculeuse'' écrite 
par Muller, et qui, en plus d'être en gpl et fortement maintenue (dernière 
version date d'il y a 2 semaines), elle nous paraît très performante et d'une 
rare adéquation.  C'est donc avec beaucoup de tristesse dans l’âme (:p), qu'on n'
utilise qu'une fraction de ses possibilités.  C'était tout l'avantage de notre 
application (et de son architecture client-server) et ce n'est pas utilisé a son
full potentiel.  C'est donc la première amérioration que devra subir notre
application, car tout est en place pour pouvoir en tirer un avantage certain.

\subsection{Droits}
Un des buts de cette architecture était de pouvoir avoir plusieurs comptes liés à
un projet, avec des droits différents. Imaginons par exemple, des professeurs qui pourraient
 éditer leurs désidérata, ou la possibilité pour un utilisateur de pouvoir suivre l'évolution d'un horaire, sans pour autant pouvoir modifier celui-ci.

\subsection{Mode hors ligne}
Nous nous basons entièrement sur le local storage, les requêtes vers le serveur ne 
servent qu'a se connecter, charger le projet et enregistrer des modifications.
Actuellement, il est possible de continuer la création d'un horaire chargé en 
mémoire sans la nécessité d'une connexion, cependant, pour pouvoir pleinement 
utiliser cette fonctionnalité, il est nécessaire de rajouter certaines fonctionnalités, telles 
qu'une resynchronisation des modifications faites en local avec la base de données lorsque la 
connexion est retrouvée. La possibilité, pour l'utilisateur, de reprendre un projet toujours conservé dans la mémoire cache.

\subsection{Upload à partir d'une base de données}
le fichier actuellement utilisé pour l'upload des attributions est le résultat d'une requête SQL. Dans un souci de clarté et de facilité pour l'utilisateur, il est plus évident de devoir rentrer un fichier manuellement étant donné qu'un autre fichier des locaux doit être également fourni. Pour qu'un upload puisse se faire de manière complète, il faudrait que la liste des locaux avec ce que ceci comporte soit également enregistrée dans une base de données afin de ne pas perdre l'utilisateur sur deux méthodes différentes d'envoi et de pouvoir ainsi garder une certaine cohérence du logiciel.

\subsection{Modifications de données}
Actuellement il n'est pas possible de modifier toutes les données intervenant 
dans l'application (notamment la création d'un prof, local, etc.) Il faudrait donc avoir la possibilité d'ajouter certaines informations directement via l'interface du logiciel, afin de prévenir de tout oubli, ou rajout de dernière minute sur le nombre de classes, etc.

\subsection{Upload de contraintes}
L'idéal serait de permettre à l'utilisateur d'uploader son fichier de contraintes (sous forme de fichier XML par exemple), ou de pouvoir les entrer manuellement via l'interface. N'étant pas l'objectif de base de notre travail de fin d'étude, celui-ci étant plus orienté sur la programmation par contraintes à proprement parlé et l'intégration de cette programmation dans un outil d'aide à la création d'horaire, et donc l'élaboration de ce logiciel, ceci limite l'utilisateur de l'application aux uniques configurations que nous aurions choisis au préalable. Il faudrait donc dans l'idéal permettre à l'utilisateur de pouvoir manipuler lui même les contraintes.
